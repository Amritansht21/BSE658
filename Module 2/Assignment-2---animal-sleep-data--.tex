% Options for packages loaded elsewhere
\PassOptionsToPackage{unicode}{hyperref}
\PassOptionsToPackage{hyphens}{url}
%
\documentclass[
]{article}
\usepackage{amsmath,amssymb}
\usepackage{iftex}
\ifPDFTeX
  \usepackage[T1]{fontenc}
  \usepackage[utf8]{inputenc}
  \usepackage{textcomp} % provide euro and other symbols
\else % if luatex or xetex
  \usepackage{unicode-math} % this also loads fontspec
  \defaultfontfeatures{Scale=MatchLowercase}
  \defaultfontfeatures[\rmfamily]{Ligatures=TeX,Scale=1}
\fi
\usepackage{lmodern}
\ifPDFTeX\else
  % xetex/luatex font selection
\fi
% Use upquote if available, for straight quotes in verbatim environments
\IfFileExists{upquote.sty}{\usepackage{upquote}}{}
\IfFileExists{microtype.sty}{% use microtype if available
  \usepackage[]{microtype}
  \UseMicrotypeSet[protrusion]{basicmath} % disable protrusion for tt fonts
}{}
\makeatletter
\@ifundefined{KOMAClassName}{% if non-KOMA class
  \IfFileExists{parskip.sty}{%
    \usepackage{parskip}
  }{% else
    \setlength{\parindent}{0pt}
    \setlength{\parskip}{6pt plus 2pt minus 1pt}}
}{% if KOMA class
  \KOMAoptions{parskip=half}}
\makeatother
\usepackage{xcolor}
\usepackage[margin=1in]{geometry}
\usepackage{color}
\usepackage{fancyvrb}
\newcommand{\VerbBar}{|}
\newcommand{\VERB}{\Verb[commandchars=\\\{\}]}
\DefineVerbatimEnvironment{Highlighting}{Verbatim}{commandchars=\\\{\}}
% Add ',fontsize=\small' for more characters per line
\usepackage{framed}
\definecolor{shadecolor}{RGB}{248,248,248}
\newenvironment{Shaded}{\begin{snugshade}}{\end{snugshade}}
\newcommand{\AlertTok}[1]{\textcolor[rgb]{0.94,0.16,0.16}{#1}}
\newcommand{\AnnotationTok}[1]{\textcolor[rgb]{0.56,0.35,0.01}{\textbf{\textit{#1}}}}
\newcommand{\AttributeTok}[1]{\textcolor[rgb]{0.13,0.29,0.53}{#1}}
\newcommand{\BaseNTok}[1]{\textcolor[rgb]{0.00,0.00,0.81}{#1}}
\newcommand{\BuiltInTok}[1]{#1}
\newcommand{\CharTok}[1]{\textcolor[rgb]{0.31,0.60,0.02}{#1}}
\newcommand{\CommentTok}[1]{\textcolor[rgb]{0.56,0.35,0.01}{\textit{#1}}}
\newcommand{\CommentVarTok}[1]{\textcolor[rgb]{0.56,0.35,0.01}{\textbf{\textit{#1}}}}
\newcommand{\ConstantTok}[1]{\textcolor[rgb]{0.56,0.35,0.01}{#1}}
\newcommand{\ControlFlowTok}[1]{\textcolor[rgb]{0.13,0.29,0.53}{\textbf{#1}}}
\newcommand{\DataTypeTok}[1]{\textcolor[rgb]{0.13,0.29,0.53}{#1}}
\newcommand{\DecValTok}[1]{\textcolor[rgb]{0.00,0.00,0.81}{#1}}
\newcommand{\DocumentationTok}[1]{\textcolor[rgb]{0.56,0.35,0.01}{\textbf{\textit{#1}}}}
\newcommand{\ErrorTok}[1]{\textcolor[rgb]{0.64,0.00,0.00}{\textbf{#1}}}
\newcommand{\ExtensionTok}[1]{#1}
\newcommand{\FloatTok}[1]{\textcolor[rgb]{0.00,0.00,0.81}{#1}}
\newcommand{\FunctionTok}[1]{\textcolor[rgb]{0.13,0.29,0.53}{\textbf{#1}}}
\newcommand{\ImportTok}[1]{#1}
\newcommand{\InformationTok}[1]{\textcolor[rgb]{0.56,0.35,0.01}{\textbf{\textit{#1}}}}
\newcommand{\KeywordTok}[1]{\textcolor[rgb]{0.13,0.29,0.53}{\textbf{#1}}}
\newcommand{\NormalTok}[1]{#1}
\newcommand{\OperatorTok}[1]{\textcolor[rgb]{0.81,0.36,0.00}{\textbf{#1}}}
\newcommand{\OtherTok}[1]{\textcolor[rgb]{0.56,0.35,0.01}{#1}}
\newcommand{\PreprocessorTok}[1]{\textcolor[rgb]{0.56,0.35,0.01}{\textit{#1}}}
\newcommand{\RegionMarkerTok}[1]{#1}
\newcommand{\SpecialCharTok}[1]{\textcolor[rgb]{0.81,0.36,0.00}{\textbf{#1}}}
\newcommand{\SpecialStringTok}[1]{\textcolor[rgb]{0.31,0.60,0.02}{#1}}
\newcommand{\StringTok}[1]{\textcolor[rgb]{0.31,0.60,0.02}{#1}}
\newcommand{\VariableTok}[1]{\textcolor[rgb]{0.00,0.00,0.00}{#1}}
\newcommand{\VerbatimStringTok}[1]{\textcolor[rgb]{0.31,0.60,0.02}{#1}}
\newcommand{\WarningTok}[1]{\textcolor[rgb]{0.56,0.35,0.01}{\textbf{\textit{#1}}}}
\usepackage{graphicx}
\makeatletter
\def\maxwidth{\ifdim\Gin@nat@width>\linewidth\linewidth\else\Gin@nat@width\fi}
\def\maxheight{\ifdim\Gin@nat@height>\textheight\textheight\else\Gin@nat@height\fi}
\makeatother
% Scale images if necessary, so that they will not overflow the page
% margins by default, and it is still possible to overwrite the defaults
% using explicit options in \includegraphics[width, height, ...]{}
\setkeys{Gin}{width=\maxwidth,height=\maxheight,keepaspectratio}
% Set default figure placement to htbp
\makeatletter
\def\fps@figure{htbp}
\makeatother
\setlength{\emergencystretch}{3em} % prevent overfull lines
\providecommand{\tightlist}{%
  \setlength{\itemsep}{0pt}\setlength{\parskip}{0pt}}
\setcounter{secnumdepth}{-\maxdimen} % remove section numbering
\ifLuaTeX
  \usepackage{selnolig}  % disable illegal ligatures
\fi
\usepackage{bookmark}
\IfFileExists{xurl.sty}{\usepackage{xurl}}{} % add URL line breaks if available
\urlstyle{same}
\hypersetup{
  hidelinks,
  pdfcreator={LaTeX via pandoc}}

\author{}
\date{\vspace{-2.5em}}

\begin{document}

ASSIGNMENT 2 : DATA WRANGLING

\begin{Shaded}
\begin{Highlighting}[]
\FunctionTok{library}\NormalTok{(tidyverse)}
\end{Highlighting}
\end{Shaded}

\begin{verbatim}
## -- Attaching core tidyverse packages ------------------------ tidyverse 2.0.0 --
## v dplyr     1.1.4     v readr     2.1.5
## v forcats   1.0.0     v stringr   1.5.1
## v ggplot2   3.5.1     v tibble    3.2.1
## v lubridate 1.9.3     v tidyr     1.3.1
## v purrr     1.0.2     
## -- Conflicts ------------------------------------------ tidyverse_conflicts() --
## x dplyr::filter() masks stats::filter()
## x dplyr::lag()    masks stats::lag()
## i Use the conflicted package (<http://conflicted.r-lib.org/>) to force all conflicts to become errors
\end{verbatim}

\begin{Shaded}
\begin{Highlighting}[]
\FunctionTok{library}\NormalTok{(dplyr)}
\FunctionTok{library}\NormalTok{(tibble)}
\FunctionTok{library}\NormalTok{(ggplot2)}
\end{Highlighting}
\end{Shaded}

loading the data

\begin{Shaded}
\begin{Highlighting}[]
\FunctionTok{data}\NormalTok{(}\StringTok{"msleep"}\NormalTok{)}
\FunctionTok{head}\NormalTok{(msleep)}
\end{Highlighting}
\end{Shaded}

\begin{verbatim}
## # A tibble: 6 x 11
##   name    genus vore  order conservation sleep_total sleep_rem sleep_cycle awake
##   <chr>   <chr> <chr> <chr> <chr>              <dbl>     <dbl>       <dbl> <dbl>
## 1 Cheetah Acin~ carni Carn~ lc                  12.1      NA        NA      11.9
## 2 Owl mo~ Aotus omni  Prim~ <NA>                17         1.8      NA       7  
## 3 Mounta~ Aplo~ herbi Rode~ nt                  14.4       2.4      NA       9.6
## 4 Greate~ Blar~ omni  Sori~ lc                  14.9       2.3       0.133   9.1
## 5 Cow     Bos   herbi Arti~ domesticated         4         0.7       0.667  20  
## 6 Three-~ Brad~ herbi Pilo~ <NA>                14.4       2.2       0.767   9.6
## # i 2 more variables: brainwt <dbl>, bodywt <dbl>
\end{verbatim}

\begin{Shaded}
\begin{Highlighting}[]
\CommentTok{\# converting the data to tibble}
\NormalTok{sleep\_data }\OtherTok{\textless{}{-}} \FunctionTok{as\_tibble}\NormalTok{(msleep)}
\NormalTok{sleep\_data}
\end{Highlighting}
\end{Shaded}

\begin{verbatim}
## # A tibble: 83 x 11
##    name   genus vore  order conservation sleep_total sleep_rem sleep_cycle awake
##    <chr>  <chr> <chr> <chr> <chr>              <dbl>     <dbl>       <dbl> <dbl>
##  1 Cheet~ Acin~ carni Carn~ lc                  12.1      NA        NA      11.9
##  2 Owl m~ Aotus omni  Prim~ <NA>                17         1.8      NA       7  
##  3 Mount~ Aplo~ herbi Rode~ nt                  14.4       2.4      NA       9.6
##  4 Great~ Blar~ omni  Sori~ lc                  14.9       2.3       0.133   9.1
##  5 Cow    Bos   herbi Arti~ domesticated         4         0.7       0.667  20  
##  6 Three~ Brad~ herbi Pilo~ <NA>                14.4       2.2       0.767   9.6
##  7 North~ Call~ carni Carn~ vu                   8.7       1.4       0.383  15.3
##  8 Vespe~ Calo~ <NA>  Rode~ <NA>                 7        NA        NA      17  
##  9 Dog    Canis carni Carn~ domesticated        10.1       2.9       0.333  13.9
## 10 Roe d~ Capr~ herbi Arti~ lc                   3        NA        NA      21  
## # i 73 more rows
## # i 2 more variables: brainwt <dbl>, bodywt <dbl>
\end{verbatim}

filtering out data to relevant coloumns

\begin{Shaded}
\begin{Highlighting}[]
\FunctionTok{select}\NormalTok{(sleep\_data, }\SpecialCharTok{{-}}\FunctionTok{c}\NormalTok{( genus, order))}
\end{Highlighting}
\end{Shaded}

\begin{verbatim}
## # A tibble: 83 x 9
##    name      vore  conservation sleep_total sleep_rem sleep_cycle awake  brainwt
##    <chr>     <chr> <chr>              <dbl>     <dbl>       <dbl> <dbl>    <dbl>
##  1 Cheetah   carni lc                  12.1      NA        NA      11.9 NA      
##  2 Owl monk~ omni  <NA>                17         1.8      NA       7    0.0155 
##  3 Mountain~ herbi nt                  14.4       2.4      NA       9.6 NA      
##  4 Greater ~ omni  lc                  14.9       2.3       0.133   9.1  0.00029
##  5 Cow       herbi domesticated         4         0.7       0.667  20    0.423  
##  6 Three-to~ herbi <NA>                14.4       2.2       0.767   9.6 NA      
##  7 Northern~ carni vu                   8.7       1.4       0.383  15.3 NA      
##  8 Vesper m~ <NA>  <NA>                 7        NA        NA      17   NA      
##  9 Dog       carni domesticated        10.1       2.9       0.333  13.9  0.07   
## 10 Roe deer  herbi lc                   3        NA        NA      21    0.0982 
## # i 73 more rows
## # i 1 more variable: bodywt <dbl>
\end{verbatim}

\begin{Shaded}
\begin{Highlighting}[]
\NormalTok{sleep }\OtherTok{\textless{}{-}} \FunctionTok{select}\NormalTok{(sleep\_data, }\SpecialCharTok{{-}}\FunctionTok{c}\NormalTok{( genus, order))}
\end{Highlighting}
\end{Shaded}

\begin{Shaded}
\begin{Highlighting}[]
\NormalTok{sleep }\OtherTok{\textless{}{-}} \FunctionTok{mutate}\NormalTok{(sleep , }\AttributeTok{prop =}\NormalTok{ ((brainwt}\SpecialCharTok{/}\NormalTok{bodywt)}\SpecialCharTok{*}\DecValTok{100}\NormalTok{))}
\NormalTok{sleep}
\end{Highlighting}
\end{Shaded}

\begin{verbatim}
## # A tibble: 83 x 10
##    name      vore  conservation sleep_total sleep_rem sleep_cycle awake  brainwt
##    <chr>     <chr> <chr>              <dbl>     <dbl>       <dbl> <dbl>    <dbl>
##  1 Cheetah   carni lc                  12.1      NA        NA      11.9 NA      
##  2 Owl monk~ omni  <NA>                17         1.8      NA       7    0.0155 
##  3 Mountain~ herbi nt                  14.4       2.4      NA       9.6 NA      
##  4 Greater ~ omni  lc                  14.9       2.3       0.133   9.1  0.00029
##  5 Cow       herbi domesticated         4         0.7       0.667  20    0.423  
##  6 Three-to~ herbi <NA>                14.4       2.2       0.767   9.6 NA      
##  7 Northern~ carni vu                   8.7       1.4       0.383  15.3 NA      
##  8 Vesper m~ <NA>  <NA>                 7        NA        NA      17   NA      
##  9 Dog       carni domesticated        10.1       2.9       0.333  13.9  0.07   
## 10 Roe deer  herbi lc                   3        NA        NA      21    0.0982 
## # i 73 more rows
## # i 2 more variables: bodywt <dbl>, prop <dbl>
\end{verbatim}

\begin{Shaded}
\begin{Highlighting}[]
\NormalTok{sleep }\OtherTok{\textless{}{-}} \FunctionTok{arrange}\NormalTok{(sleep, sleep\_total)}
\end{Highlighting}
\end{Shaded}

ASKING QUESTIONS :

Q1.Does diet affect sleep time ?

\begin{Shaded}
\begin{Highlighting}[]
\FunctionTok{ggplot}\NormalTok{(sleep, }\FunctionTok{aes}\NormalTok{(}\AttributeTok{x =}\NormalTok{ vore , }\AttributeTok{y =}\NormalTok{ sleep\_total , }\AttributeTok{color =}\NormalTok{ vore, }\AttributeTok{alpha =} \FloatTok{0.2}\NormalTok{)) }\SpecialCharTok{+} \FunctionTok{geom\_point}\NormalTok{() }\SpecialCharTok{+} \FunctionTok{geom\_boxplot}\NormalTok{()}
\end{Highlighting}
\end{Shaded}

\includegraphics{Assignment-2---animal-sleep-data--_files/figure-latex/unnamed-chunk-7-1.pdf}
We can see that insectivores have a higer total sleep than carnivore,
herbivore or omnivore , who seem to have nearly similar average sleep.

\begin{Shaded}
\begin{Highlighting}[]
\FunctionTok{ggplot}\NormalTok{( }\AttributeTok{data =}\NormalTok{ sleep , }\FunctionTok{aes}\NormalTok{(}\AttributeTok{x=}\NormalTok{ sleep\_total))  }\SpecialCharTok{+} \FunctionTok{geom\_histogram}\NormalTok{( }\AttributeTok{bins =} \DecValTok{16}\NormalTok{) }\SpecialCharTok{+} \FunctionTok{facet\_wrap}\NormalTok{(}\SpecialCharTok{\textasciitilde{}}\NormalTok{vore , }\AttributeTok{nrow =} \DecValTok{2}\NormalTok{)}
\end{Highlighting}
\end{Shaded}

\includegraphics{Assignment-2---animal-sleep-data--_files/figure-latex/unnamed-chunk-8-1.pdf}
we see that in carnivores there is nearly even distribution of total
sleep hours, while the omnivores seem to be all close by in a single
peak and herbivores with two distinct peaks.

Q2. what is the effect of Body wt. on sleep

\begin{Shaded}
\begin{Highlighting}[]
\CommentTok{\# sleep plotted against weight, the clusters were magnified by limiting a maximum body weight.}
\FunctionTok{ggplot}\NormalTok{( sleep , }\FunctionTok{aes}\NormalTok{(}\AttributeTok{x =}\NormalTok{ bodywt, }\AttributeTok{y =}\NormalTok{ sleep\_total, }\AttributeTok{color =}\NormalTok{ vore)) }\SpecialCharTok{+} \FunctionTok{geom\_point}\NormalTok{()}
\end{Highlighting}
\end{Shaded}

\includegraphics{Assignment-2---animal-sleep-data--_files/figure-latex/unnamed-chunk-9-1.pdf}

\begin{Shaded}
\begin{Highlighting}[]
\FunctionTok{ggplot}\NormalTok{( }\FunctionTok{filter}\NormalTok{(sleep,bodywt }\SpecialCharTok{\textless{}} \DecValTok{2000}\NormalTok{) , }\FunctionTok{aes}\NormalTok{(}\AttributeTok{x =}\NormalTok{ bodywt, }\AttributeTok{y =}\NormalTok{ sleep\_total, }\AttributeTok{color =}\NormalTok{ vore)) }\SpecialCharTok{+} \FunctionTok{geom\_point}\NormalTok{()}
\end{Highlighting}
\end{Shaded}

\includegraphics{Assignment-2---animal-sleep-data--_files/figure-latex/unnamed-chunk-9-2.pdf}

\begin{Shaded}
\begin{Highlighting}[]
\FunctionTok{ggplot}\NormalTok{( }\FunctionTok{filter}\NormalTok{(sleep,bodywt }\SpecialCharTok{\textless{}} \DecValTok{200}\NormalTok{) , }\FunctionTok{aes}\NormalTok{(}\AttributeTok{x =}\NormalTok{ bodywt, }\AttributeTok{y =}\NormalTok{ sleep\_total, }\AttributeTok{color =}\NormalTok{ vore)) }\SpecialCharTok{+} \FunctionTok{geom\_point}\NormalTok{()}
\end{Highlighting}
\end{Shaded}

\includegraphics{Assignment-2---animal-sleep-data--_files/figure-latex/unnamed-chunk-9-3.pdf}

\begin{Shaded}
\begin{Highlighting}[]
\FunctionTok{ggplot}\NormalTok{( }\FunctionTok{filter}\NormalTok{(sleep,bodywt }\SpecialCharTok{\textless{}} \DecValTok{100}\NormalTok{) , }\FunctionTok{aes}\NormalTok{(}\AttributeTok{x =}\NormalTok{ bodywt, }\AttributeTok{y =}\NormalTok{ sleep\_total, }\AttributeTok{color =}\NormalTok{ vore)) }\SpecialCharTok{+} \FunctionTok{geom\_point}\NormalTok{()}
\end{Highlighting}
\end{Shaded}

\includegraphics{Assignment-2---animal-sleep-data--_files/figure-latex/unnamed-chunk-9-4.pdf}

\begin{Shaded}
\begin{Highlighting}[]
\FunctionTok{ggplot}\NormalTok{( }\FunctionTok{filter}\NormalTok{(sleep,bodywt }\SpecialCharTok{\textless{}} \DecValTok{50}\NormalTok{) , }\FunctionTok{aes}\NormalTok{(}\AttributeTok{x =}\NormalTok{ bodywt, }\AttributeTok{y =}\NormalTok{ sleep\_total, }\AttributeTok{color =}\NormalTok{ vore)) }\SpecialCharTok{+} \FunctionTok{geom\_point}\NormalTok{()}
\end{Highlighting}
\end{Shaded}

\includegraphics{Assignment-2---animal-sleep-data--_files/figure-latex/unnamed-chunk-9-5.pdf}

\begin{Shaded}
\begin{Highlighting}[]
\FunctionTok{ggplot}\NormalTok{( }\FunctionTok{filter}\NormalTok{(sleep,bodywt }\SpecialCharTok{\textless{}} \DecValTok{25}\NormalTok{) , }\FunctionTok{aes}\NormalTok{(}\AttributeTok{x =}\NormalTok{ bodywt, }\AttributeTok{y =}\NormalTok{ sleep\_total, }\AttributeTok{color =}\NormalTok{ vore)) }\SpecialCharTok{+} \FunctionTok{geom\_point}\NormalTok{()}
\end{Highlighting}
\end{Shaded}

\includegraphics{Assignment-2---animal-sleep-data--_files/figure-latex/unnamed-chunk-9-6.pdf}

\begin{Shaded}
\begin{Highlighting}[]
\FunctionTok{ggplot}\NormalTok{( }\FunctionTok{filter}\NormalTok{(sleep,bodywt }\SpecialCharTok{\textless{}} \DecValTok{10}\NormalTok{) , }\FunctionTok{aes}\NormalTok{(}\AttributeTok{x =}\NormalTok{ bodywt, }\AttributeTok{y =}\NormalTok{ sleep\_total, }\AttributeTok{color =}\NormalTok{ vore)) }\SpecialCharTok{+} \FunctionTok{geom\_point}\NormalTok{()}
\end{Highlighting}
\end{Shaded}

\includegraphics{Assignment-2---animal-sleep-data--_files/figure-latex/unnamed-chunk-9-7.pdf}

\begin{Shaded}
\begin{Highlighting}[]
\FunctionTok{ggplot}\NormalTok{( }\FunctionTok{filter}\NormalTok{(sleep,bodywt }\SpecialCharTok{\textless{}} \FloatTok{2.5}\NormalTok{) , }\FunctionTok{aes}\NormalTok{(}\AttributeTok{x =}\NormalTok{ bodywt, }\AttributeTok{y =}\NormalTok{ sleep\_total, }\AttributeTok{color =}\NormalTok{ vore)) }\SpecialCharTok{+} \FunctionTok{geom\_point}\NormalTok{()}
\end{Highlighting}
\end{Shaded}

\includegraphics{Assignment-2---animal-sleep-data--_files/figure-latex/unnamed-chunk-9-8.pdf}

\begin{Shaded}
\begin{Highlighting}[]
\FunctionTok{ggplot}\NormalTok{( }\FunctionTok{filter}\NormalTok{(sleep,bodywt }\SpecialCharTok{\textless{}} \FloatTok{0.5}\NormalTok{) , }\FunctionTok{aes}\NormalTok{(}\AttributeTok{x =}\NormalTok{ bodywt, }\AttributeTok{y =}\NormalTok{ sleep\_total, }\AttributeTok{color =}\NormalTok{ vore)) }\SpecialCharTok{+} \FunctionTok{geom\_point}\NormalTok{()}
\end{Highlighting}
\end{Shaded}

\includegraphics{Assignment-2---animal-sleep-data--_files/figure-latex/unnamed-chunk-9-9.pdf}
I plotted dot plots based on total sleep vs body wt. and dies. We can
clearly see that animals with lower body weight sleep more. this could
also explain the higher average sleep in insectivores in prvious study
as all insectivore were redents of small body weight and thus slept
more.

this cal also be seen in the head and tail of sleep, the heaviest
sleepers are the lighest animals !!

\begin{Shaded}
\begin{Highlighting}[]
\FunctionTok{head}\NormalTok{(}\FunctionTok{select}\NormalTok{(sleep, name, sleep\_total, bodywt ))}
\end{Highlighting}
\end{Shaded}

\begin{verbatim}
## # A tibble: 6 x 3
##   name             sleep_total bodywt
##   <chr>                  <dbl>  <dbl>
## 1 Giraffe                  1.9  900. 
## 2 Pilot whale              2.7  800  
## 3 Horse                    2.9  521  
## 4 Roe deer                 3     14.8
## 5 Donkey                   3.1  187  
## 6 African elephant         3.3 6654
\end{verbatim}

\begin{Shaded}
\begin{Highlighting}[]
\FunctionTok{tail}\NormalTok{(}\FunctionTok{select}\NormalTok{(sleep, name, sleep\_total, bodywt ))}
\end{Highlighting}
\end{Shaded}

\begin{verbatim}
## # A tibble: 6 x 3
##   name                   sleep_total bodywt
##   <chr>                        <dbl>  <dbl>
## 1 Long-nosed armadillo          17.4  3.5  
## 2 North American Opossum        18    1.7  
## 3 Giant armadillo               18.1 60    
## 4 Thick-tailed opposum          19.4  0.37 
## 5 Big brown bat                 19.7  0.023
## 6 Little brown bat              19.9  0.01
\end{verbatim}

Q3. Does conservation have any effect on sleep duration

\begin{Shaded}
\begin{Highlighting}[]
\FunctionTok{ggplot}\NormalTok{(sleep, }\FunctionTok{aes}\NormalTok{(}\AttributeTok{x =}\NormalTok{ conservation , }\AttributeTok{y =}\NormalTok{ sleep\_total , }\AttributeTok{color =}\NormalTok{ vore , }\AttributeTok{alpha =} \FloatTok{0.6}\NormalTok{)) }\SpecialCharTok{+} \FunctionTok{geom\_point}\NormalTok{() }\SpecialCharTok{+} \FunctionTok{geom\_boxplot}\NormalTok{()}
\end{Highlighting}
\end{Shaded}

\includegraphics{Assignment-2---animal-sleep-data--_files/figure-latex/unnamed-chunk-11-1.pdf}

\begin{Shaded}
\begin{Highlighting}[]
\FunctionTok{ggplot}\NormalTok{( sleep , }\FunctionTok{aes}\NormalTok{(}\AttributeTok{x =}\NormalTok{ vore, }\AttributeTok{y =}\NormalTok{ sleep\_total, }\AttributeTok{color =}\NormalTok{ conservation)) }\SpecialCharTok{+} \FunctionTok{geom\_point}\NormalTok{()}
\end{Highlighting}
\end{Shaded}

\includegraphics{Assignment-2---animal-sleep-data--_files/figure-latex/unnamed-chunk-12-1.pdf}
We can cleraly see that across all diet types, the sleep time of Least
concerned (lc) and Non-threathened (nc) type animals is higher as
compared to Critically endangered (cd), Endangered (en) and
Vulnerable(vu) type animals.

there could be many reasons as to why we see this factor: my hypothesis
is their environment of constant danger due to being less in number
forced them to be more alert.

\end{document}
